%%
%% The following code sets up the document formatting
%%

%this assumes that res_yy.sty is in some path
\documentstyle[hyperref, margin, centered]{res}

\hypersetup{backref,pdfpagemode=Full,backref}

\addtolength{\oddsidemargin}{-0.45in}
\addtolength{\voffset}{-0.30in}
\addtolength{\textwidth}{1.00in} \addtolength{\textheight}{1.50in}

\renewcommand{\namefont}{\LARGE\bf}

\newcommand{\sectiontitle}{\sc}


%%
%% The following code defines some macros for terms which have raised font
%% (ie 4\fourth would result 4th with the 'th' raised (superscripted)
%%

\def\Cplusplus{{\rm C\raise.5ex\hbox{\small ++}}}
\def\CSharp{{\rm C\raise.5ex\hbox{\small \#}}}
% 'st' 'nd' 'rd' 'th' superscripts for numbers
\def\first{{\raise.5ex\hbox{\small st}}}
\def\second{{\raise.5ex\hbox{\small nd}}}
\def\third{{\raise.5ex\hbox{\small rd}}}
\def\fourth{{\raise.5ex\hbox{\small th}}}



%%
%% starting the actual document
%%

\begin{document}

%the name in big fonts at the top of resume
%this is left aligned
\name{Yiqing (Hannah) Xue}

%this is right aligned
\address{
http://hanax.co \textbar\ yx348@cornell.edu \textbar\ +1 (646) 245-7152
}

\begin{resume}

\vspace{25pt}

\section{\sectiontitle{About}}

Product engineer with a passion for art and design.

\section{\sectiontitle{Education}}

\textbf{Cornell Tech}, New York, NY \hfill 2015--2017 \\
\textsl{M.S. in Information Systems, Connective Media with Merit Scholarship} \\
Selected Courses: Applied Machine Learning, Future Interaction
Technologies, Master Thesis on Autonomous Vehicles UX.

\textbf{Peking University}, Beijing, China \hfill 2011--2015 \\ 
\textsl{B.S. in Computer Science with Honors and Scholarships}\\
Received 2014 Google China Anita Borg Scholarship \\
Selected Courses: Computer Networks (Honor Track), Operating Systems (Honor Track), Compilers, Human-Computer Interaction.
% , Computer Architecture, Software Engineering

\textbf{Peking University}, Beijing, China \hfill 2012--2015 \\ 
\textsl{B.S. in Psychology (dual degree)}\\
Selected Courses: Social Psychology, Cognitive Psychology, Experimental Psychology.

\begin{formatb}
  \employer{l}\dates{r}\\
  \title{l}\\
  % \location{l}
  \body\\
\end{formatb}


\section{\sectiontitle{Employment}}

\employer{\textbf{Airbnb}, San Francisco, CA}
\title{\textsl{Software Engineer} (React, React Native, Ruby on Rails)}
\dates{2017 -- present}
\begin{position}
Full-stack development at Airbnb Experiences team. 
\end{position}

\employer{\textbf{Marvel Prototyping}, London, UK}
\title{\textsl{Product Engineering Intern}}
\dates{2017}
\begin{position}
Work on multiple features on Canvas team to enable online prototype creation, making design accessible for all.
\end{position}

\employer{\textbf{Airbnb}, San Francisco, CA}
\title{\textsl{Software Engineering Intern} (React, Ruby on Rails)}
\dates{2015}
\begin{position}
Web development at Host Growth team. 
Brought the first photo editing tool to all Airbnb hosts.
\end{position}

\employer{\textbf{Rhizome at the New Museum}, New York, NY}
\title{\textsl{Design Intern} (Sketch, user study, prototyping)}
\dates{2016}
\begin{position}
Worked as part of the webrecorder team to help preserve highly dynamic digital art on the Web. \\
Mainly worked on visual branding. Iterated on visual identity guidelines including logo and UI components.
\end{position}

\employer{\textbf{Google}, Mountain View, CA}
\title{\textsl{Software Engineering Intern} (Google Web Toolkit, Android)}
\dates{2014}
\begin{position}
Enabled advertisers to copy multiple ads asynchronously from existing ad groups at a large scale.

\end{position}


%%
%% We use the same formatting for projects as for work experience
%% Shown below is the formatting used previously
%%
%%  \begin{formatb}
%%    \employer{l}\title{r}\\
%%    \location{l}\dates{r}\\
%%    \body\\
%%  \end{formatb}
%%
%% 
%%  Note that \location is now being used for non-location information
%%


\begin{formatb}
  \employer{l}\dates{r}\\
  \body\\
\end{formatb}

\section{\sectiontitle{Research}}

\employer{\textbf{Baidu Research}, Beijing, China}
\dates{2015}
\begin{position}
Designed and built a natural user interface for a confidential smart home hardware at Baidu Institute of Deep-Learning.
\end{position}

\employer{\textbf{Microsoft Research Asia}, Beijing, China}
\dates{2013-2014}
\begin{position}
Making 2D barcode scalable for wireless visual communication. \\
Collaborated with Yale University. Paper accepted by ACM MobiCom 2014 (Acceptance Rate 16.4\%).
\end{position}

\section{\sectiontitle{Skills}}
JavaScript (React + Flux, React Native). Ruby on Rails. Java. \\
UX design, visual design. Prototyping. User study.

\section{\sectiontitle{Interests}}
Modern art. Fashion (spent a summer at Parsons). Tai-chi (certified National Athlete).

\end{resume}
\end{document}
